\chapter{Conclusion}
\label{Conclusion}

The goal of this project was to create a system that uses IPFS to explore blockchain. To achieve this, I needed to create my own decentralized and distributed database system on the top of IPFS that supports queries and indexes. Feeder and ExplorerCore communicates with this database system to store (Feeder) or retrieve data (ExplorerCore). To visualize data in GUI, ExplorerGUI can be used, and for obtaining data with RestApi there is ExplorerAPI.

The research section of this project focuses mainly on the IPFS principles (content-based addressing, nodes linking, etc.). In the research section, there are also briefly discussed selected cryptocurrencies for exploring. Great effort was devoted to designing the whole system and creating a functional prototype.

In the next semester, I want to highly improve database system by implementing fluent query API (enabling concatenation for function \texttt{select}, \texttt{where}, \texttt{orderBy}, \texttt{limit}, etc.) and support logical operators (\texttt{and}, \texttt{or}). I want to research and compare more indexing strategies and make performance testing and profiling. Also comparasion with BlockBook in the same conditions would be interesting. There is high potential by using IPFS in this system to create new use cases that no other system for exploring cryptocurrencies blockchain supports. For example, with some heuristic, I believe, that with this system, we can follow crypto coins after exchange for another crypto coins (thanks to linking between objects in IPLD layer).

