\chapter{Conclusion}
\label{Conclusion}
The goal of this thesis was to create a platform that uses IPFS to explore blockchain. To achieve this, we needed to create our own decentralized and distributed database system that supports advanced queries and indexes on top of the IPFS. Our database system has optimized synchronization, and as proved in Chapter \ref{benchmark}, it is faster with every connected peer. Thanks to language choice (Typescript), Explorer-core module in which database is implemented, works in a browser environment and desktop (via node.js). We created several applications using this decentralized database, namely Feeder that connects to the blockchain data source and stores it to IPFS, ExplorerGUI is used as a presentation layer for users, and ExplorerAPI provides a simple interface for integrating with other applications.

The research section of this project focuses mainly on the IPFS principles (content-based addressing, object linking, and others). Selected cryptocurrencies for exploring are also briefly discussed. Considerable effort was devoted to designing the whole platform and creating a functional prototype.

Although the task of this work was to create a simple blockchain explorer, this project aroused great interest in the IPFS community, because our database system allows developers to create decentralized serverless applications with a relatively efficient database solution. Thanks to this work, it is possible to host entire information systems in IPFS. There are many benefits to using our database system for developers. The main advantages include the cost of hosting, as in our solution with a higher number of users, the load on the central peer decreases, as well as the availability of data even in the event of a failure of part of the peers. Our Query system is well adapted to today ORM frameworks and uses fluent query language and supports multiple conditions in a single query. 

In the future, it is possible to improve Query system by supporting more operators (\texttt{like}, \texttt{groupBy}). Other data connectors as Blockbook can be added to the Feeder (for example direct connection to the blockchain via full node). Other index structures can be implemented for better performance in some queries (for example trie\footnote{https://en.wikipedia.org/wiki/Trie} for prefix search).



 